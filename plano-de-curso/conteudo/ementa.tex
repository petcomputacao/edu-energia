\section{Ementa}
Os participantes do curso irão aprender a coletar, classificar, visualizar e analisar dados de consumo de energia elétrica do Smart Campus UFCG. Para tanto, serão ofertados os seguintes conteúdos:

\begin{enumerate}
	\item Introdução à lógica de programação, à linguagem Python e ao ambiente Google Colab
		\begin{enumerate}
			\item Ambiente Google Colab
			\item Conceito de algoritmo
			\item Variáveis e tipos de dados
			\item Operadores aritméticos, lógicos e de comparação
			\item Estruturas de controle (if, elif, else, for e while)
			\item Estruturas de dados essenciais (listas, tuplas, dicionários e conjuntos)
			\item Funções
			\item Gráficos com Matplotlib
		\end{enumerate}
	\item Introdução à eletricidade
		\begin{enumerate}
			\item Grandezas fundamentais em eletricidade
			\item Potência elétrica (aparente, ativa e reativa)
			\item Fator de potência
			\item Tensão elétrica
			\item Corrente elétrica
			\item Sinal
			\item Magnitude média de um sinal alternado
			\item Eficiência energética
			\item Modalidades tarifárias e indicadores úteis de consumo de energia elétrica
		\end{enumerate}
	\item Análise descritiva de dados
		\begin{enumerate}
			\item Tabelas e gráficos
			\begin{enumerate}
				\item Coleta e armazenamento de dados
				\item Tipos de variáveis
				\item Estudando a distribuição de frequências em uma variável
				\item Variáveis qualitativas - nominais e ordinais
				\item Variáveis quantitativas discretas
				\item Variáveis quantitativas contínuas
				\item Distribuição de frequências
				\item Gráfico para séries temporais
				\item Diagrama de dispersão
			\end{enumerate}
			\item Síntese numérica
			\begin{enumerate}
				\item Medidas de tendência central (média aritmética simples, mediana e moda)
				\item Medidas de variabilidade (amplitude total, desvio padrão, coeficiente de variação e regra de desvio padrão para distribuições simétricas)
				\item Medidas de posição (percentis e escores padronizados)
				\item Boxplot
				\item Comparação gráfica de conjunto de dados
			\end{enumerate}
		\end{enumerate}
\end{enumerate}
